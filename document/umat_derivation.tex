% !TEX program = xelatex
\documentclass{article}
\usepackage[UTF8]{ctex}
\usepackage[margin=1in]{geometry}
\usepackage{float}
\usepackage{amsfonts,amssymb} 
\usepackage{bm}
\usepackage{amsmath}
\usepackage{color}

\title{Neo Hookean UMAT}
\author{徐辉}
\begin{document}
\maketitle
\section{理论}
Abaqus 指标映射关系
\begin{table}[H]
    \centering
    \begin{tabular}{ll}
        \hline
        $(i,j)$ & $a$ \\
        \hline
        (1,1)   & 1   \\
        (2,2)   & 2   \\
        (3,3)   & 3   \\
        (1,2)   & 4   \\
        (1,3)   & 5   \\
        (2,3)   & 6   \\
        \hline
    \end{tabular}
    \caption{Abaqus 指标映射关系}
\end{table}
线性理论的Lame常数: $\lambda_0$和$\mu_0$
\begin{equation*}
    \begin{aligned}
        \lambda & = \lambda_0               \\
        \mu     & = \mu_0 - \lambda_0 \ln J \\
    \end{aligned}
\end{equation*}
变形梯度
$$\bm{F}=\frac{\partial \bm{x}}{ \partial \bm{X}}=\left(
    \begin{array}{ccc}
            F_{11} & F_{12} & F_{13} \\
            F_{21} & F_{22} & F_{23} \\
            F_{31} & F_{32} & F_{33} \\
        \end{array}
    \right)$$

左Cauchy Green变形张量
$$\bm{B}= \bm{F} \cdot \bm{F}^{\rm T}=\left(
    \begin{array}{ccc}
            F_{11}^2+F_{12}^2+F_{13}^2 & F_{11}F_{21}+F_{12}F_{22}+F_{13}F_{23} & F_{11}F_{31}+F_{12}F_{32}+F_{13}F_{33} \\
                                       & F_{21}^2+F_{22}^2+F_{23}^2             & F_{21}F_{31}+F_{22}F_{32}+F_{23}F_{33} \\
            {\rm sym}                  &                                        & F_{31}^2+F_{32}^2+F_{33}^2             \\
        \end{array}
    \right)$$
\begin{equation*}
    \begin{aligned}
        B[1]=B_{11} & =F_{11}^2+F_{12}^2+F_{13}^2             \\
        B[2]=B_{22} & =F_{21}^2+F_{22}^2+F_{23}^2             \\
        B[3]=B_{33} & =F_{31}^2+F_{32}^2+F_{33}^2             \\
        B[4]=B_{12} & =F_{11}F_{21}+F_{12}F_{22}+F_{13}F_{23} \\
        B[5]=B_{13} & =F_{11}F_{31}+F_{12}F_{32}+F_{13}F_{33} \\
        B[6]=B_{23} & =F_{21}F_{31}+F_{22}F_{32}+F_{23}F_{33} \\
    \end{aligned}
\end{equation*}

应变能函数\cite{bely}(5.4.54)
$$W(\bm{C})= \frac{1}{2}\mu_0({\rm trace}(\bm{C})-3)-\mu_0 \ln J + \frac{1}{2}\lambda_0(\ln J)^2$$


Kirchhoff应力\cite{bely}(5.4.55)
$$\bm{\tau}=\lambda_0\ln J\bm{I}+\mu_0(\bm{B}-\bm{I})$$

Cauchy 应力
$$\bm{\sigma}=\frac{\bm{\tau}}{J}=\frac{\mu_0}{J}\bm{B}-\frac{\mu_0-\lambda_0\ln J}{J}\bm{I}$$
\begin{equation*}
    \begin{aligned}
        \sigma[1]=\sigma_{11} & =\frac{\mu_0}{J}B[1]-\frac{\mu_0-\lambda_0\ln J}{J} \\
        \sigma[2]=\sigma_{11} & =\frac{\mu_0}{J}B[2]-\frac{\mu_0-\lambda_0\ln J}{J} \\
        \sigma[3]=\sigma_{11} & =\frac{\mu_0}{J}B[3]-\frac{\mu_0-\lambda_0\ln J}{J} \\
        \sigma[4]=\sigma_{12} & =\frac{\mu_0}{J}B[4]                                \\
        \sigma[5]=\sigma_{13} & =\frac{\mu_0}{J}B[5]                                \\
        \sigma[6]=\sigma_{23} & =\frac{\mu_0}{J}B[6]                                \\
    \end{aligned}
\end{equation*}

材料切线刚度可以表示为\cite{bely}(5.4.57)
\begin{equation*}
    \mathbb{C}^{\tau}_{ijkl}=\lambda \delta_{ij} \delta_{kl}+\mu(\delta_{ik}\delta_{jl}+\delta_{il}\delta_{kj})
\end{equation*}
有切线刚度的转化关系\cite{bely}(5.4.42)(5.4.46)(5.4.51)
\begin{equation*}
    \begin{aligned}
        \mathbb{C}^{SE}_{ijkl}       & =\frac{\partial S_{ij}}{\partial E_{kl}} \\
        \mathbb{C}^{\tau}_{ijkl}     & =F_{im}F_{jn}F_{kp}F_{lq}C^{SE}_{mnpq}   \\
        \mathbb{C}^{\sigma T}_{ijkl} & =\frac{1}{J}\mathbb{C}^{\tau}_{ijkl}     \\
    \end{aligned}
\end{equation*}

Abaqus切线刚度可以表示\cite{nguy}(36)为
\begin{equation*}
    \begin{aligned}
        {\rm DDSDDE}_{{\color{red}iqph}}= & \frac{1}{J}F_{iM}F_{qI}F_{pK}F_{hL}\frac{\partial S_{MI}}{\partial E_{KL}}+\frac{1}{2}(\sigma_{ip}\delta_{qh}+\sigma_{qh}\delta_{ip}+\sigma_{ih}\delta_{qp}+\sigma_{qp}\delta_{ih}) \\
    \end{aligned}
\end{equation*}
{\color{red} 注意该公式和论文中不同,认为论文中公式存在笔误,不同之处用红色标出。}指标变换并带入材料刚度有
\begin{equation*}
    \begin{aligned}
        {\rm DDSDDE}_{ijkl} & =\mathbb{C}^{\sigma T}_{ijkl} +\frac{1}{2}(\sigma_{ik}\delta_{jl}+\sigma_{jl}\delta_{ik}+\sigma_{il}\delta_{jk}+\sigma_{jk}\delta_{il})                                                                   \\
                            & =\frac{1}{J}(\lambda \delta_{ij} \delta_{kl}+\mu(\delta_{ik}\delta_{jl}+\delta_{il}\delta_{kj}))+\frac{1}{2}(\sigma_{ik}\delta_{jl}+\sigma_{jl}\delta_{ik}+\sigma_{il}\delta_{jk}+\sigma_{jk}\delta_{il})
    \end{aligned}
\end{equation*}
根据表1,写成矩阵形式有
\begin{equation}
\begin{aligned}
    {\rm DDSDDE}&=\left(
    \begin{array}{cccccc}
            (\lambda + 2\mu)/J & \lambda/J          & \lambda/J          & 0     & 0     & 0     \\
                               & (\lambda + 2\mu)/J & \lambda/J          & 0     & 0     & 0     \\
                               &                    & (\lambda + 2\mu)/J & 0     & 0     & 0     \\
                               &                    &                    & \mu/J & 0     & 0     \\
                               & {\rm sym}          &                    &       & \mu/J & 0     \\
                               &                    &                    &       &       & \mu/J \\
        \end{array}
    \right)\\
    &+
    \left(
    \begin{array}{cccccc}
            2\sigma[1] & 0            & 0          & \sigma[4]               & \sigma[5]               & 0                       \\
                       & 2\sigma[2] 0 & 0          & \sigma[4]               & 0                       & \sigma[6]               \\
                       &              & 2\sigma[3] & 0                       & \sigma[5]               & \sigma[6]               \\
                       &              &            & (\sigma[1]+\sigma[2])/2 & \sigma[6]/2             & \sigma[5]/2             \\
                       & {\rm sym}    &            &                         & (\sigma[1]+\sigma[3])/2 & \sigma[4]/2             \\
                       &              &            &                         &                         & (\sigma[2]+\sigma[3])/2 \\
        \end{array}
    \right)
\end{aligned}
\label{mmme}
\end{equation}

\section{数值模拟}
对如下问题进行数值模拟。

平面应变问题,一个1x1的块体两侧受压,边界条件为位移载荷。左侧为0.25,右侧为-0.25。

材料参数$\mu_0=2$,$\lambda_0=1000000$


\subsection{理论解}
变形梯度:
\begin{equation*}
    \bm{F}=\left(
    \begin{array}{ccc}
            0.5 & 0 & 0 \\
            0   & 2 & 0 \\
            0   & 0 & 1 \\
        \end{array}
    \right)
\end{equation*}

左Cauchy-Green变形张量:
\begin{equation*}
    \bm{B}=\bm{F}\cdot\bm{F}^{\rm T}=\left(
    \begin{array}{ccc}
            0.25 & 0 & 0 \\
            0    & 4 & 0 \\
            0    & 0 & 1 \\
        \end{array}
    \right)
\end{equation*}

Cauchy应力
\begin{equation*}
    \bm{\sigma}=-p\bm{I}+2\left(\frac{\partial W}{\partial I}\bm{B}\right)=\mu_0\bm{B}-p\bm{I}=\mu_0\left(
    \begin{array}{ccc}
            0.25-p & 0   & 0   \\
            0      & 4-p & 0   \\
            0      & 0   & 1-p \\
        \end{array}
    \right)
\end{equation*}
根据边界条件
\begin{equation*}
    \sigma_{22}=0 \quad \Rightarrow \quad p=4
\end{equation*}
因此Cauchy应力
\begin{equation*}
    \bm{\sigma}=-p\bm{I}+2\left(\frac{\partial W}{\partial I}\bm{B}\right)=\mu_0\bm{B}-p\bm{I}=\mu_0\left(
    \begin{array}{ccc}
            -3.75 & 0 & 0  \\
            0     & 0 & 0  \\
            0     & 0 & -3 \\
        \end{array}
    \right)
    =\left(
    \begin{array}{ccc}
            -7.5 & 0 & 0  \\
            0    & 0 & 0  \\
            0    & 0 & -6 \\
        \end{array}
    \right)
\end{equation*}

\subsection{模拟结果}

\subsubsection{使用$\mu$}
在公式(\ref{mmme})中,直接使用$\mu$。即不作修改,直接使用公式(\ref{mmme})。

数值模拟应力大小和解析结果相同。

第一个负特征值出现在增量步94,时间0.8358,名义应变大约为$0.5 \times 0.8358/1=41.79\%$。

\subsubsection{使用$\mu_0$}
在切线刚度的部分就认为材料不可压缩,即在公式(\ref{mmme})中,使用$\mu_0$,$\mu$出现的的地方使用$\mu_0$代替。

第一个负特征值出现在增量步10,时间0.00758, 名义应变大约为$0.5 \times 0.00758/1=0.00379\%$。

在增量步54,时间0.3877,名义应变大约为$0.5 \times 0.3877/1=19\%$时计算不收敛。

\begin{thebibliography}{1}
    \bibitem{bely}  Nonlinear Finite Elements for Continua and Structures @ Belytschko
    \bibitem{nguy} Nguyen N , Waas A M . Nonlinear, finite deformation, finite element analysis[J]. Zeitschrift Für Angewandte Mathematik Und Physik, 2016, 67(3):35.
\end{thebibliography}

\end{document}